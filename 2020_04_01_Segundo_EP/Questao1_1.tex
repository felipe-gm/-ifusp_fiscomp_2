\documentclass[12pt,runningheads]{article}
\usepackage[utf8]{inputenc}
\usepackage{amsmath,amssymb,hyperref,array,xcolor,multicol,verbatim,mathpazo}
\usepackage[normalem]{ulem}
\usepackage[pdftex]{graphicx}
\begin{document}

%%%% In most cases you won't need to edit anything above this line %%%%

\title{Segundo EP}
\author{Felipe Miyazato - 8944453}
\maketitle

\section{Questão 1}
\subsection{Equação do movimento para \(\vec{r}_1\)}
Igualando a segunda equação de Newton para o corpo 1,
\[\vec{F}_1 = m_1 \frac{d^2\vec{r}_1}{dt^2}\]
e sua lei da gravitação universal,
\[\vec{F}_1 = G m_1 m_i \frac{\vec{r}_i - \vec{r}_1}{|\vec{r}_i - \vec{r}_1|^3}\]
obtemos
\[m_1 \frac{d^2\vec{r}_1}{dt^2} = \sum_i G m_1 m_i \frac{\vec{r}_i - \vec{r}_1}{|\vec{r}_i - \vec{r}_1|^3}\]
Simplificando para o problema dos 3 corpos
\[\frac{d^2\vec{r}_1}{dt^2} = -G m_2 \frac{\vec{r_1} - \vec{r_2}}{|\vec{r_1} - \vec{r_2}|^3} - G m_3 \frac{\vec{r_1} - \vec{r_3}}{|\vec{r_1} - \vec{r_3}|^3}\]
\newpage

\subsection{Equações análogas para \(\vec{r}_2\) e \(\vec{r}_3\)}
Analogamente
\[\frac{d^2\vec{r}_2}{dt^2} = -G m_1 \frac{\vec{r_2} - \vec{r_1}}{|\vec{r_2} - \vec{r_1}|^3} - G m_3 \frac{\vec{r_2} - \vec{r_3}}{|\vec{r_2} - \vec{r_3}|^3}\]
\[\frac{d^2\vec{r}_3}{dt^2} = -G m_1 \frac{\vec{r_3} - \vec{r_1}}{|\vec{r_3} - \vec{r_1}|^3} - G m_2 \frac{\vec{r_3} - \vec{r_2}}{|\vec{r_3} - \vec{r_2}|^3}\]

\subsection{Conversão para sistema de primeira ordem}
Definindo
\[\vec{v}_i = \frac{d\vec{r}_i}{dt}\]
obtemos
\[\frac{d\vec{r}_1}{dt} = \vec{v}_1\]
\[\frac{d\vec{v}_1}{dt} = -G m_2 \frac{\vec{r_1} - \vec{r_2}}{|\vec{r_1} - \vec{r_2}|^3} - G m_3 \frac{\vec{r_1} - \vec{r_3}}{|\vec{r_1} - \vec{r_3}|^3}\]
\[\frac{d\vec{r}_2}{dt} = \vec{v}_2\]

\[\frac{d\vec{v}_2}{dt} = -G m_1 \frac{\vec{r_2} - \vec{r_1}}{|\vec{r_2} - \vec{r_1}|^3} - G m_3 \frac{\vec{r_2} - \vec{r_3}}{|\vec{r_2} - \vec{r_3}|^3}\]
\[\frac{d\vec{r}_3}{dt} = \vec{v}_3\]
\[\frac{d\vec{v}_3}{dt} = -G m_1 \frac{\vec{r_3} - \vec{r_1}}{|\vec{r_3} - \vec{r_1}|^3} - G m_2 \frac{\vec{r_3} - \vec{r_2}}{|\vec{r_3} - \vec{r_2}|^3}\]

\newpage

\section{Questão 2}
Não vamos usar as estimativas de erro da integração numérica para as variáveis de velocidade, pois é uma variável introduzida em nossa solução apenas para generalizar os métodos de integração numérica de equações de diferença de primeira ordem para ordens superiores.

As simetrias herdadas da teoria de mecânica clássica nos convidam a usar a norma euclideana para medir estimativas dos erros para a trajetória de cada corpo, assim implementou-se.

\end{document}